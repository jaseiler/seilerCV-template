\documentclass[11pt]{seilercv} 

\setcounter{page}{1}

\begin{document}

\header{Jennifer}{Seiler}{PhD}{Computational Physicist}
\begin{aside}
  \section{Personal}
%Date of Birth: May 5th, 1983\\
\bf{Location: }\hfill Washington, DC
\bf{Place of Birth}: \hfill Washington, DC
\bf{Citizenship: } \hfill United States 
%Marital Status: Unmarried \\
\bf{First Language:  } \hfill English
%   \bf{Country: }\hfill  United States
  \section{Contact}
   %\href{mailto:jenn.seiler@gggmmmaill.com}{jenn.seiler@gggmmmail.com}\hfill \faEnvelope 
   %{1(571111)228-0438}\hfill\faPhone
   \href{http://jennseiler.com}{jennseiler.com}\hfill \faHome
   \href{https://github.com/jaseiler}{github.com/jaseiler} \hfill \faGithubAlt 
   \href{https://www.linkedin.com/in/jennseiler/}{linkedin.com/in/jennseiler/}\hfill  \faLinkedin
   \bf{GScholar: } \hfill \href{https://scholar.google.com/citations?user=I9vG6PAAAAAJ}{rebrand.ly/seilergs}
   \bf{ORCID: } \hfill \href{https://orcid.org/0000-0003-2855-3945}{0003-2855-3945}
   %\bf{: } \hfill \href{}{}, 
  \section{Programming}
  C\#, C/C++, Python, \textsc{Fortran, Pascal}, R, Java, JavaScript, JQuery, Unity, XML, SQL, SML, Basic, Perl, HTML, \LaTeX{}, Aduino, Octave, MATLAB, sed, lisp,  CSS3 \& HTML5\ldots
  \section{Keywords}
  Computational Physics, 
  Numerical Simulation, 
  High Performance Computing, 
  Finite Element Analysis, 
  Statistical Methods, 
  Reproducibility Practices,
  Gravitational Astrophysics, 
  Discrete mathematics,
  Scientific computing, 
  Mathematical modeling, 
  Data Science\ldots
  \section{Languages}
  English (native) 
  German (basic) 
  Spanish (basic)
\end{aside}
\setlength{\tabcolsep}{0pt}

%\addresses
%{
%Computational Astrophysicist\\
%\textit{Giant Army}\\
%Washington, DC %85745 USA
%Astrophysical Sciences Division Code 663 \\
%NASA/Goddard Space Flight Center \\
%8800 Greenbelt Road \\
%Greenbelt, MD 20771 USA \\
%}
%
%{
%Office: 1 (301) 286-3454 \\
%\href{http://jennseiler.com}{\texttt{jennseiler.com}} \\
%}
%

%\begin{llist}
\begin{textblock}{9.6}(5.25, 1.63)
\section{Summary}
I am a computational physicist and game developer looking for a stimulating new challenge. For the past 17 years I have worked on various large scale computational modeling and simulation projects. I have also worked in developing and promoting reproducible research and analysis methods, and maintainable shared code. Although my degrees are in physics, I have very strong and unique computer science background in software development and testing, numerical simulations, analysis, database management, and game design.
I like hunting dragons and I get compulsively excited about results on the horizon. I am never bored. 

\section{Education}
\vspace{-0.3em}\employer{Max Planck Institute For Gravitational Physics\hfill\textit{Ph.D., magna cum laude}\\ %{Max-Planck-Institut f\"{u}r Gravitationsphysik}
  \sf{Potsdam, Germany}}
\location {Aug 2005 - Feb 2010}
\vspace{-20pt}
\begin{tabbing}
\textbf{Thesis Title:}\hspace{1.7cm} \= \=\textit{\href{http://jennseiler.com/docs/thesis.pdf}{Numerical Simulations of Binary Black Hole Spacetimes}} \\ 
\>\>\textit{\href{http://jennseiler.com/docs/thesis.pdf}{and a Novel Approach to Outer Boundaries}}\\
\textbf{Thesis Advisers:} \> \>Luciano Rezzolla \& Bernard Schutz \\
\textbf{University Affiliation:} \>Gottfried Leibniz Universit\"{a}t Hannover\\
\> via the International Max Plank Research School Fellowship\\
\end{tabbing}
\vspace{-15pt}
\employer{Cornell University  \hfill \textit{B.A., Honors Physics} \\
\sf{Ithaca, NY}}
\location{Aug 2001 - May 2005}
\vspace{-20pt}
\begin{tabbing}
\textbf{Adviser:} \hspace{.9in} \=Saul Teukolsky\\
\textbf{Research Emphasis:} \>Computational Physics; Numerical Relativity\\\
\end{tabbing}
\vspace{-10pt}
\employer{Hayfield Secondary School\hfill\textit{H.S., Honors A.P.} \\ \sf{Alexandria, VA}} 
\location{Sept 1997 - June 2001}
\textbf{Research Emphasis:} \hspace{.2in} Physics; Architecture; Computer Science \\
%\vspace{-0.6cm}\sectiontitle{Keywords} numerical relativity; gravitational astrophysics; computational physics; discrete mathematics; high performance computing; massively parallel, highly scalable systems; scientific computing; distributed computing; large scale mathematical modeling; data processing and analysis

\vspace{-0.3cm}\section{Fellowships, Grants, and Awards}
\begin{tabbing}
\vspace{-0.3em}{\textbullet~  Visiting Scientist Grant from Universitat de les Illes Balears for October 2010}
\\{\textbullet~ \href{http://www.slac.stanford.edu/spires/topcites/2009/eprints/to_gr-qc_annual.shtml}{Three of the top cited GR papers of 2009}}
\\{\textbullet~ \href{http://arxiv.org/pdf/0709.0942}{James Hartle Award}:} \textit{Constraint Preserving Boundaries in 2nd Order Form} at GRG18
%\\{\href{http://arxiv.org/pdf/0709.0942}{\hspace{.2in}\textit{http://arxiv.org/pdf/0709.0942}}}
\\{\textbullet~ NASA NY Space Grant 2003:} For work under Saul Teukolsky on DUSTVis
\\{\textbullet~ FermiLab Internships for Physics Majors 2002:} BTeV Trigger Algorithm
\\{\textbullet~ US DoD's SEAP (Science and Engineering Apprenticeship Program) 2001}
\\{\textbullet~} \={Award of Recognition of Outstanding Achievement 2001, Solid State Physics}
\\ \>{from the US Naval Research Laboratory}
\\{\textbullet~ Treasurer for the Cornell Chapter of Society of Physics Students 2003-2005}
\\{\textbullet~ Recognition from Nat. Science Teachers Association 2000}
\\{\textbullet~ Recognition from Graduate Women in Science 2000, Theoretical Physics }
\\ \>{(for \textit{Acoustic Thermometry of Sea Water}) }
\\{\textbullet~ Intel Science Talent Search Semifinalist 2001 (for \textit{Longitudinal Flow in Au-Au Collisions})} 
%\\{\href{http://www.sciserv.org/sts/60sts/semi_VA.asp}{\hspace{.2in}\textit{http://www.sciserv.org/sts/60sts/semi\_{VA.asp}}}}
\\{\textbullet~ University of Southern California Young Scientist of the Year 2000 }
\\{\textbullet~ CIA Outstanding Young Scientist (for \textit{Acoustic Thermometry of Sea Water}) }
\\{\textbullet~ Armed Forces and Communications and Electronics scholarship summer of '99 }
\\{\textbullet~ Award of Recognition: Society of Women Engineers (\textit{Acoustic Thermometry of Sea Water}) }
%\\{Intel Virginia State Science Talent Search 2nd place (for \textit{Longitudinal Flow in Au-Au Collisions})}
%\\{Grand Prize in the 1999 NOVA Intel Science and Engineering Fair}
%\\{Honors Science Program at Michigan State University in the National Superconducting Cyclotron}
\\{\textbullet~ Physlink.com Young Scientist of the Year 2000 (for Acoustic Thermometry of Sea Water)}
\end{tabbing}
%\\{\href{http://web.archive.org/web/20000816230028/www.physlink.com/ysaward2000_press.cfm}{\hspace{.2in}\textit{http://www.physlink.com/ysaward2000\_{press.cfm}}}}}
\end{textblock}
\pagebreak

\section{Research and Professional Experience}
\job{Giant Army}{April 2015 - Current}{I work as Developer and Staff Astrophysicist on improving the physics, planetary science, climate simulation, and stellar astronomy of \href{http://universesandbox.com}{Universe Sandbox}. The game is a physics-based space simulator that allows users to simulate galaxies, planetary systems, climates, collisions, structure formation, and much more.}
\job{Columbia University}{February 2013 - October 2014}{Postdoctoral position in the Department of Statistics researching issues of reproducibility in science. A major focus was \href{http://researchcompendia.science}{ResearchCompendia.science}. ResearchCompendia is a web service that allows researchers to run codes associated with scientific publications. The service allows authors of publications to create companion websites on which others may reproduce the paper's results or to run their own parameters.}
\job{NASA Goddard Space Flight Center}{March 2010 -July 2012}{NASA Postdoctoral Position (NPP) in the numerical relativity group for the \href{https://lisa.nasa.gov/}{LISA project}.  I wrote numerical simulations of binary black hole spacetimes, electromagnetic counterparts to black hole interactions, and matter fields around binary black hole systems.}
\job{Max-Planck Institut f\"{u}r Gravitationsphysik}{June 2005 - February 2010}{PhD work on numerical simulations of black hole spacetimes.  My focus was on well-posed constraint preserving boundary conditions.  With additional work on constraint damping methods, gravitational wave detectability, and phenomenological waveforms and predictions for merged binary final spin and recoil velocity.}
\job{Albert Einstein Institute}{June 2004 - August 2004}{Visiting scientist in Potsdam, Germany.  I wrote a parallelised numerical code to generate initial data and evolve a simulation of the propagation of gravitational waves off a potential in a three dimensional coordinate system and track constraint propagation and violation. }
\job{Cornell University}{November 2002 - October 2004}{Worked for Prof. Saul Teukolsky on \href{https://www.black-holes.org/code/SpEC.html}{software for the visualization} and analysis of numerical simulations of solutions to the Einstein equations. These included inspiraling neutron stars and black holes systems, binary black holes, and accretion disks.   }
%Called DUSTVis, it is an OpenDX visualization program designed to be used with the Caltech/Cornell DUST algorithm.
\job{Cornell University}{December 2004 - May 2004}{Designed software for an industrial chemical waste exchange program, titled the National Trash to Treasure Network, for submission to the EPA as a project for voluntary participation offered to companies as an alternative to fines. }
%A learning algorithm finds other chemicals with similar properties for the buyer and evaluates the cost of transportation and processing. 
\job{Fermi National Accelerator Lab}{May 2002-August 2002}{Participated in the Internship for Physics Majors Program (IPM).  I designed and programmed the track-finding algorithm for the Level 1 Trigger Code for the BTeV project. After finding tracks, it looks for detached tracks which signify an exotic decay, on-the-fly in the detector firmware. }
\job{Naval Research Laboratories}{June 2001 - Sept. 2001}{Worked in the Electronics Science \& Technology Division on the optimization of natural growth of Silicon dioxide, SiGe, and SiC samples via Molecular Beam Epitaxy.  Experimented with the temperature and surface segregation dependencies of Phosphorous doping rates via MBE. }
\job{Michigan State University}{May 2000 -  August 2000}{I worked in the National Superconducting Cyclotron at Michigan State University. I wrote data analysis code in C++ and did the analysis of data collected of Au on Au collisions at energies from 20-60 AMeV for a better understanding the equation of state for stellar core collapses.}

\pagebreak
\section{Publications}
{ {L. Rezzolla, P. Diener, E. N. Dorband, D. Pollney, C. Reisswig, E. Schnetter, {\bf J. Seiler}.}  \textbf{The Final Spin From the Coalescence of Aligned-spin Black-hole Binaries}.  \textit{Astrophys.\ J.\  {\bf 674} (2008) L29}.  
Preprint:   \href{http://arxiv.org/abs/0710.3345}{arXiv.org:0710.3345} [gr-qc]}\vspace{0.2cm} \\*
{ {L.~Rezzolla, E.~Barausse, E.~N.~Dorband, D.~Pollney, C.~Reisswig, {\bf J.~Seiler} and S.~Husa}. \textbf{On the final spin from the coalescence of two black holes}.  \textit{Phys.\ Rev.\  D {\bf 78} (2008) 044002}. \\ Preprint: \href{http://arxiv.org/abs/0712.3541}{arXiv:0712.3541}[gr-qc] \vspace{0.2cm} \\
{{\bf J.~Seiler}, B.~Szilagyi, D.~Pollney}.  \textbf{Constraint Preserving Boundaries for a Generalized Harmonic Evolution Systems}.  \textit{Class.\ Quant.\ Grav.\  {\bf 25} (2008) 175020}. Preprint: \href{http://arxiv.org/abs/0802.3341}{arXiv:0802.3341} [gr-qc]\vspace{0.2cm} \\*
{B.~Aylott, {\it et al.}~(including {\bf J.~Seiler})}. \textbf{Testing gravitational-wave searches with numerical relativity waveforms:
Results from the first Numerical INJection Analysis (NINJA) project}.
\textit{Class. \ Quant. \ Grav. \ {\bf 26} (2009) 165008}. Preprint: \href{http://arxiv.org/abs/0901.4399}{arXiv:0901.4399} [gr-qc]. \vspace{0.2cm} \\
{B.~Aylott, {\it et al.}(including {\bf J.~Seiler})}. \textbf{Status of NINJA: the Numerical INJection Analysis project}.  \textit{Class.\ Quant.\ Grav.\ {\bf 26} (2009) 114008}. Preprint: \href{http://arxiv.org/abs/0905.4227}{arXiv:0905.4227} [gr-qc] \vspace{0.2cm} \\
{C.~Reisswig, S.~Husa, L.~Rezzolla, E.~Dorband, D.~Pollney and {\bf J.~Seiler}}.
\textbf{Gravitational-wave detectability of equal-mass black-hole binaries with aligned spins}. \textit{Phys.\ Rev.\  D {\bf 80} (2009) 124026}. Preprint: \href{http://arxiv.org/abs/0907.0462}{arXiv:0907.0462} [gr-qc] \vspace{0.2cm} \\*
{L.~Santamaria, F.~Ohme, P.~Ajith, B.~Bruegmann, N.~Dorband, M.~Hannam, S.~Husa, P.~Moesta, D.~Pollney, C.~Reisswig, E.~L.~Robinson, {\bf J.~Seiler}, B. Krishnan. }\textbf{Matching post-Newtonian and numerical relativity waveforms: systematic errors and a new phenomenological model for non-precessing black hole binaries}
 \textit{Phys.\ Rev.\  D {\bf 82} (2010) 064016}.  Preprint: \href{http://arxiv.org/abs/1005.3306}{arXiv:1005.3306} [gr-qc]\vspace{0.2cm} \\*
{P.~Ajith, M.~Hannam, S.~Husa, Y.~Chen, B. Bruegmann, N. Dorband, D. Muller, F. Ohme, D. Pollney, C. Reisswig, L. Santamaria, {\bf J.~Seiler}.} \textbf{``Complete'' gravitational-waveforms for black-hole binaries with non-precessing spins}.  \textit{Phys. Rev. Lett. {\bf 106} (2011) 241101} Preprint: \href{http://arxiv.org/abs/0909.2867}{arXiv:0909.2867} [gr-qc]\vspace{0.2cm} \\*
{V.~Stodden, S.~Miguez, {\bf J.~Seiler}.} \textbf{ResearchCompendia.org: Cyberinfrastructure for Reproducibility and Collaboration in Computational Science}. \textit{IEEE Computing in Science \& Engineering {\bf 17(1)} (2015) 12-19}. Access: \href{http://online.qmags.com/CISE0115?pg=14&mode=2#pg14&mode2?fs=2&pg=14&mode=2}{Scientific Software Communities}\vspace{0.2cm} \\*
{V.~Stodden, {\bf J.~Seiler}, Z.~Ma. }\textbf{An empirical analysis of journal policy effectiveness for computational reproducibility}. \textit{Proceedings of the National Academy of Sciences {\bf 115.11} (2018): 2584-2589.}}%\vspace{-.25cm} }

\section{Manuscripts} 
{\small {J. Seiler, J. Baker, B. Kelly} \textbf{ Precession Mapping of Black Hole Binaries via Minimization of Asymmetric Harmonic Modes of Gravitational Waves }[scrapped, scooped, March 2012] \vspace{0.1cm}\\*
{J. Seiler, D. Pollney, B. Wardell, D. Nunez} \textbf{ Constraint Preserving Boundary Conditions for BSSN in the Linearized Regime } [scrapped, scooped Jan 2012]} \vspace{0.1cm} \\
{J. Seiler, S. Husa, J. Baker} \textbf{ Numerical Simulation of Black Hole Binaries with `Trumpet' Initial Data } [July 2012] 
%\pagebreak
\section{Further Technical Skills }
{%{{ \textbf{Languages:} \hspace{.9in} English (native), German (basic), Spanish (basic) }} \vspace{0.07cm} \\
{{ \textbf{Operating Systems:} \hspace{.34in} Linux (preferred), Mac (current), Unix, DOS, Windows }} \vspace{0.07cm} \\
%{{ \textbf{Programming/Scripting:} \hspace{.01in}  C\#, C/C++, Python, \textsc{Fortran, Pascal}, Java, PHP, XML, SQL, SML, Basic, Perl, Ruby, HTML, LaTeX, Javascript, Processing, Octave, MATLAB, sed, lisp\ldots}}\vspace{0.07cm} \\
{{ \textbf{Libraries \& Software:} \hspace{.22in} Unity, jQuery, Hadoop, NumPy/SciPy/iPython, LAPACK, HDF5, VTK, OpenDX, PBS, Globus, Scali, OpenMP, MPICH, LAM, Cactus, MATLAB, Arduino, vi, emacs, Photoshop, OpenOffice, Ableton, Flash, Mathematica, Maxima, ROOT, VisIt, Amira, PAW \ldots}} }
%{{ \textbf{Supercomputing on: } \hspace{.24in}  Peyote, Lagavulin, Belladonna, Damiana, Leibniz Computer Center (Munich), Teragrid and Louisiana Optical Network Initiative clusters, NCCS computers }}}
\section{Select Contributed Talks} \vspace{1.0em}
{ {
\textbf{``ResearchCompendia: Connecting Computation to Publication"}\\
\textit{University of Texas Austin, TX, USA} \hfill December 16, 2013 \\
Scientific Software Days \\[2ex]
\textbf{``Binary Orbital Dynamics from the Analysis of Spherical Harmonic Modes of Gravitational Waves''}\\
\textit{University of Maryland College Park, USA}\hfill November 20, 2011 \\
Gravity Theory Seminars  \\[2ex]
\textbf{``Listening to the Geometry of Spacetime: Ripples in the Fabric of the Universe"}\\
\textit{Burning Man, Black Rock City, NV}\hfill August 29, 2011 \\
Phage Talks: Institute For Higher Yearrning \\[2ex]
\textbf{``Gravitational-wave Detectability of Black-hole Binaries With Aligned Spins''}\\
\textit{NASA Goddard Space Flight Center, USA}\hfill March 26, 2010 \\
Astrophysics Sciences Division Director's Seminar  \\ [2ex]
\textbf{``Final Spin from Binary Black Hole Coalescence''}\\
\textit{Salamanca, Spain}\hfill September 19, 2008 \\ 
XXXI Spanish Relativity Meeting (E.R.E. 2008) \\[2ex]
\textbf{``From General Relativity to Black Hole Observation''}\\
\textit{Salamanca, Spain}\hfill September 19, 2008 \\ 
XXXI Spanish Relativity Meeting (E.R.E. 2008) (plenary talk)\\[2ex]
\textbf{``Final Spin from Binary Black Hole Coalescence''}\\
\textit{California Institute of Technology, Pasadena, CA, USA}\hfill August 22, 2008 \\ 
TAPIR Theoretical Astrophysics and Relativity Seminar \\[2ex]
\textbf{``2nd Order in Space Constraint Preserving Summation by Parts Boundaries''}\\
\textit{Puerto de la Cruz, Tenerife, Spain}\hfill September 10-14, 2007 \\ 
XXX Spanish Relativity Meeting (E.R.E. 2007) \\[2ex]
\textbf{``Constraint Preserving Boundary Treatment in 2nd Order Form''}\\
\textit{Sydney, Australia}\hfill July 8-14, 2007 \\
18th International Conference on General Relativity and Gravitation (GRG18) \\[2ex]
\textbf{``Boundary Treatments for the Einstein Equations in 2nd Order Form''}\\
 \textit{Palma de Mallorca, Spain} \hfill September 4-8, 2006 \\ 
XXIX Spanish Relativity Meeting (E.R.E. 2006) \\[2ex]
\textbf{``Generalised Harmonic Coordinates in 2nd Order ''}\\ 
\textit{AEI, Potsdam, Germany} \hfill November, 2005 \\
Sonder-Forschungsbereich / TransRegio 7 Video Seminars\\[2ex]
\textbf{``Generalised Harmonic Coordinates using Abigel''} \\
\textit{Oberjoch, Germany} \hfill October 10-14, 2005 \\
2005 Oberjoch Seminars }}
\pagebreak
\section{Select Workshops and Conferences}
19th International Conference on General Relativity and Gravitation (GRG19) \\ \bigskip
\textit{Mexico City, Mexico}\hfill July 5-10, 2010 \\
Numerical Relativity and Data Analysis/CAPRA Meeting (NRDA/CAPRA 2010) \\ \bigskip
\textit{Waterloo, Canada}\hfill June 20-26, 2010 \\
Numerical Relativity and Data Analysis Meeting (NRDA 2009) \\ \bigskip
\textit{Golm, Germany}\hfill July 6-9, 2009 \\
XXXI Spanish Relativity Meeting (E.R.E. 2008) \\ \bigskip
\textit{Salamanca, Spain}\hfill September 15-19, 2008 \\
Numerical Relativity and Data Analysis Meeting (NRDA 2008) \\ \bigskip
\textit{Syracuse, NY}\hfill August 11-14, 2008 \\
Frontiers in Numerical Gravitational Astrophysics (J.A. Wheeler School) \\ \bigskip
\textit{Erice, Italy}\hfill June 27-July 5, 2008 \\
Post Newton 2008 International Workshop \\ \bigskip
\textit{Jena, Germany}\hfill June 11-14, 2008 \\
XXX Spanish Relativity Meeting (E.R.E. 2007) \\ \bigskip
\textit{Puerto de la Cruz, Tenerife, Spain}\hfill September 10-14, 2007 \\
18th International Conference on General Relativity and Gravitation (GRG18) \\ \bigskip
\textit{Sydney, Australia}\hfill July 8-14, 2007 \\
AEI Performance Improvement Workshop \\ \bigskip
\textit{Albert-Einstein-Institut, Potsdam, Germany} \hfill December 4-15, 2006 \\
From Geometry to Numerics Workshop \\ \bigskip
\textit{Institut Henry Poincar\'{e}, Paris, France} \hfill November 20-24, 2006 \\
XXIX Spanish Relativity Meeting (E.R.E. 2006) \\ \bigskip
\textit{Palma de Mallorca, Spain} \hfill September 4-8, 2006 \\ 
New Frontiers in Numerical Relativity Conference \\ \bigskip
\textit{Albert-Einstein-Institut, Potsdam, Germany} \hfill July 17-21, 2006 \\
3rd High-End Visualization Workshop \\ \bigskip
\textit{University of Innsbruck, Obergurgl, Austria} \hfill April 25-28, 2006 \\
2005 Oberjoch Seminars \\
\textit{University of T\"{u}bingen, Oberjoch, Germany} \hfill October 10-14, 2005 
%
\section{Extracurricular Interests} {Coursera/EdX/Udacity classes, Physics outreach, Fire performance (poi, rope dart, staff), Open source programming, Electronics, Arduino,
Sustainability outreach, Indoor and outdoor rock climbing, Skiing, Hiking, Kayaking, Scuba, Go, Interactive multimedia installation art, Burning Man community, Vegetarian cooking} 
\pagebreak
%\end{llist}

\end{document}

%\sectiontitle{Relevant Courses}
% {Advanced Placement Comptuer Science}
%\\ {CS 211- Object Oriented Programming }
%\\ {CS 312- Functional Programming}
%\\ {Advanced Placement Introductory Mechanics.  An Intro to Mechanics by Kleppner and Kolnekov.}
%\\ {Phys 217- Honors Electricity and Magnetism. Electricity and Magnetism, Vol. 2, by Purcell.}
%\\ {Phys 218- Waves, and Optics.  }
%\\ {Phys 316- Modern Physics 1.  }
%\\ {Phys 317- Modern Physics 2.  }
%\\ {Phys 318- Honors Analytic Mechanics.  }
%\\ {Phys 327- Advanced Electricity and Magnetism. }
%\\ {Phys 341- Thermodynamics and Statistical Physics.  }
%\\ {Phys 360- Electronic Circuits Lab.  }
%\\ {Phys 410- Experimental Physics Lab. }
%\\ {Phys 443- Quantum Mechanics.  }
%\\ {Phys 444- High Energy and Particle Physics.  }
%\\ {Phys 445- Introduction to General Relativity.  }
%\\ {Phys 480- Computational Physics.  }
%\\ {Phys 490- Independent Study in Physics. Three semesters.  Black Holes and Gravitational Waves.}
%\\ {A\&EP 321- Mathematical Physics 1. }
%\\ {A\&EP 322- Mathematical Physics 2.  }
%\\ {AP Calc BC- Calculus 1 \& 2.  }
%\\ {Math 223- Honors Linear Algebra and Calculus on Manifolds 1.  }
%\\ {Math 224- Honors Linear Algebra and Calculus on Manifolds 2.  }
%\\ {Math 420- Differential Equations and Dynamical Systems.  }
%\\ {Math 454- Introduction to Differential Geometry.  }
%\\ {Astro 211- Introduction to Stars, Galaxies and Cosmology.  }
%\\ {Astro 233- Topics in Astronomy and Astrophysics.  } 
%\\ {M\&AE 459- Introduction to Controlled Fusion.  } 
%
